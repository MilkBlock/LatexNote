\documentclass[utf8]{ctexart}
\usepackage{geometry}
\geometry{a4paper,scale=0.8}
\author{Mineral}
\title{记YJF同学的一个有趣观点}
\date {\today}
\begin{document}
		\maketitle
		\section{水文一篇}
		\subsection{序}
		
		古人云:“\par 为天地立心,
		\par 为生民立命,
		\par 为往圣继绝学,
		\par 为万世开太平。”	
		今日剑锋同学曰:“为天地立心,为母人卖命。”
		听起来很有趣的样子。
		\subsection{内容}
		

		\par	不知什么时候起我察觉到了自己对于“美女”这个概念潜移默化的改变。
		\par 对于这个概念最早开始即原始社会时代是以生殖取向为标准,正如“丰乳肥臀”。
		\par 但随着社会的进步这个审美变化为了“白幼瘦”,YJF认为这是纯纯的资本主义审美
		\par 	这话有一部分道理,毕竟审美的变迁同样是社会生产力的变迁,随着物质生活的提高必然会引向一种奢侈,正如同人类如今的生理需求一部分都是广告创造出来的。浅到刷牙深至整容,无非是资本主义想方设法提高存在感的一种手段罢了。
		\par 同样白幼瘦的审美也利于资本主义提高存在感。
		\par 当然不可忽略的是人脑的物质结构天生决定了人类确实有对于存在曼妙曲线的形状存在好感,在我看来这是一种“自然的象征”,正如同人类对于完美契合的科学规律有自发的憧憬。
		\par 我们大部分人的审美一定是向“更加舒适”靠齐的,不妨让我们想想谁在那些年代拥有偏白的皮肤,谁在北美喜欢晒日光床把自己的皮肤搞成橙色,那一定是妥妥的有钱人。对的,“更加舒适”可以是嫁入豪门或者成为赘婿!
		\par 在中国这片神奇的土地上,在女性成为消费主体的国情下,我们已经可以看到一些侵蚀现象,“我负责貌美如花,你负责赚钱养家。”,这种口号中的分工模式无非是女性负责消费,男性负责赚钱。然而我们知道貌美如花正如同名画一样有欣赏价值而无实用价值,这仍然是资本主义留下的痕迹,与毛时代的“妇女能顶半边天”形成巨大的反差,资本在这改革开放的几十年间努力的取悦需求旺盛的人群——女性,并尝试用广告偶像等手段塑造更多的需求,努力将女性进一步异化为纯粹的消费工具。
		\par 显然目前男女群体裂开的形式是拜其所赐,截然不同的分工必然导致互相不理解,重创了结婚率。当然这只是次要原因,毕竟同性之间的差距才更有意思【笑】。
		\subsection{附言}
		\par 其次,YJF同学的大作《ROU YU》也很有意思,让我想起了某人归结所有欲望的源头为食欲的观点。
		\par YJF写到最初的图腾是海洋,是海洋中的盐巴,而非那似人非人的女神,这是最接近食欲的图腾
					


\end{document}
