\ifx\allfiles\undefined
\documentclass[12pt, a4paper,oneside, UTF8]{ctexbook}
\usepackage[dvipsnames]{xcolor}
\usepackage{amsmath}   % 数学公式
\usepackage{amsthm}    % 定理环境
\usepackage{amssymb}   % 更多公式符号
\usepackage{graphicx}  % 插图
\usepackage{mathrsfs}  % 数学字体
\usepackage{enumitem}  % 列表
\usepackage{geometry}  % 页面调整
\usepackage{unicode-math}
\usepackage{ulem}
\usepackage{listings}
\usepackage{xcolor}
\lstset { %
		language=C++,
		backgroundcolor=\color{black!5}, % set backgroundcolor
		basicstyle=\footnotesize,% basic font setting
}
\usepackage[colorlinks,linkcolor=black]{hyperref}

\graphicspath{ {flg/},{../flg/}, {config/}, {../config/} }  % 配置图形文件检索目录
\linespread{1.5} % 行高

% 页码设置
\geometry{top=25.4mm,bottom=25.4mm,left=20mm,right=20mm,headheight=2.17cm,headsep=4mm,footskip=12mm}

% 设置列表环境的上下间距
\setenumerate[1]{itemsep=5pt,partopsep=0pt,parsep=\parskip,topsep=5pt}
\setitemize[1]{itemsep=5pt,partopsep=0pt,parsep=\parskip,topsep=5pt}
\setdescription{itemsep=5pt,partopsep=0pt,parsep=\parskip,topsep=5pt}

% 定理环境
% ########## 定理环境 start ####################################
\theoremstyle{definition}
\newtheorem{defn}{\indent 定义}[section]

\newtheorem{lemma}{\indent 引理}[section]    % 引理 定理 推论 准则 共用一个编号计数
\newtheorem{thm}[lemma]{\indent 定理}
\newtheorem{corollary}[lemma]{\indent 推论}
\newtheorem{criterion}[lemma]{\indent 准则}

\newtheorem{proposition}{\indent 命题}[section]
\newtheorem{example}{\indent \color{SeaGreen}{例}}[section] % 绿色文字的 例 ,不需要就去除\color{SeaGreen}{}
\newtheorem*{rmk}{\indent 注}

% 两种方式定义中文的 证明 和 解 的环境:
% 缺点:\qedhere 命令将会失效【技术有限,暂时无法解决】
\renewenvironment{proof}{\par\textbf{证明.}\;}{\qed\par}
\newenvironment{solution}{\par{\textbf{解.}}\;}{\qed\par}

% 缺点:\bf 是过时命令,可以用 textb f等替代,但编译会有关于字体的警告,不过不影响使用【技术有限,暂时无法解决】
%\renewcommand{\proofname}{\indent\bf 证明}
%\newenvironment{solution}{\begin{proof}[\indent\bf 解]}{\end{proof}}
% ######### 定理环境 end  #####################################

% ↓↓↓↓↓↓↓↓↓↓↓↓↓↓↓↓↓ 以下是自定义的命令  ↓↓↓↓↓↓↓↓↓↓↓↓↓↓↓↓

% 用于调整表格的高度  使用 \hline\xrowht{25pt}
\newcommand{\xrowht}[2][0]{\addstackgap[.5\dimexpr#2\relax]{\vphantom{#1}}}

% 表格环境内长内容换行
\newcommand{\tabincell}[2]{\begin{tabular}{@{}#1@{}}#2\end{tabular}}

% 使用\linespread{1.5} 之后 cases 环境的行高也会改变,重新定义一个 ca 环境可以自动控制 cases 环境行高
\newenvironment{ca}[1][1]{\linespread{#1} \selectfont \begin{cases}}{\end{cases}}
% 和上面一样
\newenvironment{vx}[1][1]{\linespread{#1} \selectfont \begin{vmatrix}}{\end{vmatrix}}

\def\d{\textup{d}} % 直立体 d 用于微分符号 dx
\def\R{\mathbb{R}} % 实数域
\newcommand{\bs}[1]{\boldsymbol{#1}}    % 加粗,常用于向量
\newcommand{\ora}[1]{\overrightarrow{#1}} % 向量

% 数学 平行 符号
\newcommand{\pll}{\kern 0.56em/\kern -0.8em /\kern 0.56em}

% 用于空行\myspace{1} 表示空一行 填 2 表示空两行  
\newcommand{\myspace}[1]{\par\vspace{#1\baselineskip}}

\begin{document}
\chapter{2023暑假图形学与数学笔记}
\section{}
% \title{{\Huge{\textbf{标题:一个 \LaTeX 的数学笔记模板}}} \\ ———副标题:格物致知,慎思明辨}
\author{作者:到底}
\date{\today}
\maketitle                   % 在单独的标题页上生成一个标题

\thispagestyle{empty}        % 前言页面不使用页码
\begin{center}
\Huge\textbf{前言}
\end{center}

if people do not believe that mathematics is simple, 
it is only because they do not realize how complicated life is. ——John von Neumann

\begin{flushright}
\begin{tabular}{c}
\today \\ 格物致知,慎思明辨
\end{tabular}
\end{flushright}

\newpage                      % 新的一页
\pagestyle{plain}             % 设置页眉和页脚的排版方式(plain:页眉是空的,页脚只包含一个居中的页码)
\setcounter{page}{1}          % 重新定义页码从第一页开始
\pagenumbering{Roman}         % 使用大写的罗马数字作为页码
\tableofcontents              % 生成目录

\newpage                      % 以下是正文
\pagestyle{plain}
\setcounter{page}{1}          % 使用阿拉伯数字作为页码
\pagenumbering{arabic}
% \setcounter{chapter}{-1}    % 设置 -1 可作为第零章绪论从第零章开始
 % 单独编译时,其实不用编译封面目录之类的,如需要不注释这句即可
\else
\fi
%  ↓↓↓↓↓↓↓↓↓↓↓↓↓↓↓↓↓↓↓↓↓↓↓↓↓↓↓↓ 正文部分

\begin{thm}[离散函数的梯度的散度的计算方式]
梯度的散度,在数学中的表达是Laplace 算子,对于离散函数$ f $,我们有
\begin{align*}
\nabla^2 f &= \frac{f(x+\mathrm{d}x,y) + f(x-\mathrm{d}x,y) + f(x,y+\mathrm{d}y)+ f(x,y-\mathrm{d}y)  - 4f(x,y) }{\mathrm{d}x^2 }  
\end{align*}
因为根据泰勒展开我们有
\begin{align*}
f(x+\mathrm{d}x ,y) &=  f(x,y) + f'(x,y)\mathrm{d}x + \frac{f''(x,y) \mathrm{d}x^2}{2}  +\ldots\\
f(x-\mathrm{d}x ,y) &=  f(x,y) - f'(x,y)\mathrm{d}x + \frac{f''(x,y) \mathrm{d}x^2}{2}  +\ldots
\end{align*}	
两者相加即可得到
\end{thm}

\begin{thm}[$ f(\vec r) = \frac{\vec r}{r^3}$的梯度的散度 ]
		对于函数  $f(\vec r) = \frac{\vec r}{r^2}$我们有
		\begin{align*}
				\nabla^2 f(\vec r) &=  4\pi \delta (\vec r)  
		\end{align*}
		考虑在$ (0,0) $点,显然函数是未定义的。
		但我们可以从积分的层面去描绘未定义的值。
		\begin{align*}
			r \to  0 \\	
			\oiint_s \vec f \mathrm{d}   \vec a  &= \iiint_V \nabla \cdot f \mathrm{d} v   
		\end{align*}
		于是我们对于原点有有 
		\begin{align*}
			\int_{0}^{ 2\pi } \int_{0}^{ \pi } r^2\sin phi\mathrm{d}  \phi \mathrm{d}  \theta&=   4\pi 
		\end{align*}
		于是
		\begin{align*}
							\nabla ^2 f &=  4\pi \delta (\vec r)
		\end{align*}
		
\end{thm}

\begin{thm}[乘积的Operand操作数的变化对结果的影响]
		考虑 $ f(m,n)= m*n $ 在这个函数中若m和n同时改变得
\end{thm}

\begin{thm}[向量乘积的微分法则]
		\begin{align*}
				\nabla \cdot (f*\vec A) &=\nabla f\cdot \vec A+f\nabla \cdot \vec A
		\end{align*}

		而在图形学中应用拉普拉斯方程通常有
		\begin{align*}
				\nabla \cdot (\omega \nabla u) &= \nabla \omega \cdot \nabla u + \omega\nabla \cdot  \nabla u     
		\end{align*}
\end{thm}
\begin{thm}[计算万有引力势能]
	\begin{align*}
			f(x) &=  \nabla \Phi \\
			f(x) &=  -\sum_{j=1,j\neq i}^{N}\frac{ \nabla x_i (\rho_j v_j) }{4\pi \|x_i -x_j\|_2} \\
			f_i &= - \sum_{j=1,j\neq i}^{N} \frac{\rho_j v_j(x_i-x_j)}{4\pi \|x_i-x_j\|_2^3}  
	\end{align*}	
\end{thm}
\begin{thm}[伯努利不等式(体积>=边长之和-维数+1]
	\begin{align*}
			x_1 x_2\ldots x_{n} \ge x_1+x_2+\ldots+x_{n} - n+1 \\
			(x_1+1)(x_2+1)+\ldots+(x_{n}+1) \ge x_1+x_2+\ldots+x_{n} +1
	\end{align*}	
	想象一个n维空间中边长为1的正方体,如果另一个长方体完全覆盖正方体或者被这个正方体包裹,那么他的边长就满足
	体积\ge 边长之和-维数+1
\end{thm}
\begin{thm}[n维空间中边长递增的长方体,它的体积
		小于其平均边长的n北]
		一个较弱的推论是当边长从1开始且以1为增量
	\begin{align*}
			n!<(\frac{n+1}{2} )^n 
	\end{align*}	
	这揭示了指数和阶乘之间的不等式关系,
	我们直接证明对于任何起始边长和增量都满足这个条件,即证
	\begin{align*}
		k(k+c)\ldots(k+nc)\ge (\frac{2k+nc}{2})^n
	\end{align*}
	移项后只需证明
	\begin{align*}
			1\le \frac{2k+nc }{4(k+i)(k+nc-i)} 
	\end{align*}
	利用基本不等式可得
	\begin{align*}
			1\le \frac{\left( a+b \right) ^2}{4ab} 
	\end{align*}
	% $\phi  \Phi \varphi \psi \Psi$
	得证

\end{thm}
\begin{thm}[分式线性函数]
形如
\begin{align}
		\frac{ax+b}{cx+d} 
\end{align}
称之为分式线性函数
\par 求其反函数相当于是把它的四项系数重新排列
\begin{align}
	-	\frac{by-d}{cy-a} 
\end{align}
\end{thm}
\begin{thm}[求$ 2x-\left\lfloor x \right\rfloor  $的反函数]
\end{thm}
\begin{thm}[求$x$左边最近的奇数]
首先做出图像,观察到是对$ \left\lfloor x \right\rfloor $ 的$x$轴和$ y $ 轴同时应用拉伸
因此可得
\begin{align*}
		f(x) = 2\left\lfloor \frac{x+1}{2}  \right\rfloor -1
\end{align*}
\end{thm}
\begin{thm}[取整函数分析]
	考虑如下代码	
	\begin{lstlisting}
	base =(x[p]*inv_dx - 0.5).cast(int)
	fx = x[p] * inv_dx - base.cast(float)
	\end{lstlisting}
	在这里不便于利用函数的偏移分析,应该直接质问取整函数与x的差 
	w\begin{align*}
			\epsilon  &= x- \left\lfloor x-0.5\right\rfloor \\
			\epsilon &\in \left( 0.5,1.5 \right) 
	\end{align*}	
\end{thm}
\begin{thm}[B-Spline Kernels核函数]
		定义
		\begin{align*}
				P\left( u \right) =\sum_{i=0}^{n} P_iB_{i,k}(u) 
				u \in \left[ u_{k-1},u_{n+1} \right] 
		\end{align*}
		B-spline基函数求出算法应用最广泛的是deBoor-cox递推算法:
		\begin{align*}
				B_{i,k}\left( u \right) &=\frac{u-u_i}{u_{i+k-1}-u_i} *B_{i,k-1}u+\frac{u_{i+k-u}}{u_{i+k}-u_{i+1}} \\
				B_{i,1}\left( u \right) &=\left\{\begin{aligned}
								1  && u_i<u<u_{i+1}\\
	0 && Otherwise
				\end{aligned}\right.
		\end{align*}
\end{thm}
\begin{thm}[Bezier curve]
\end{thm}
\begin{thm}[How Euler compute $ \sum_{n=1}^{\infty} \frac{1}{n^2}   $]
\begin{align*}
		\sin x &= x- \frac{x^3}{3!} + \frac{x^5}{5!} -\ldots \\
		\frac{\sin x}{x} &= 1- \frac{x^2}{3!} + \frac{x^4}{5!} -\ldots \\
						 &= A[(x+\pi )(x-\pi ][(x+2\pi )(x-2\pi )]\ldots\\
		1&= A(-\pi ^2)(-4\pi ^2)(-9\pi ^2)\ldots   &&\text{代入} x=0
\end{align*}
对于$ \frac{1}{3!}  $
\end{thm}
\begin{thm}[代数基本定理]
		代数学基本定理:任何复系数一元n次多项式 方程在复数域上至少有一根(n≥1),由此推出,n次复系数多项式方程在复数域内有且只有n个根(重根按重数计算)。代数基本定理在代数乃至整个数学中起着基础作用。 据说,关于代数学基本定理的证明,现有200多种证法。
\end{thm}

\begin{thm}[质点系角动量的分解]
\begin{align*}
		L &= \sum_{i}^{} r_i \times m_i\vec v_i\\
		&= \sum_{i}^{}(\vec c+ \vec r'_i)\times m_i(\vec v_c+\vec v'_i) \\
		&= \vec r_c\times (\sum_i m_i)\vec v_c+ \vec r_c\times (\sum_im_i\vec v'_i)+(\sum_im_i\vec r'_i)\times \vec v_c+\sum_i\vec r'_i\times m_i\vec v'_i\\
		&= \vec r_c\times m\vec v_c+\vec r_c\times \vec p'_c+m\vec r'_c\times v_c+\sum_i\vec r'_i\times m_i\vec v'_i \\
		&= \vec r_c\times m\vec v_c + \sum_i\vec r'_i\times m_i\vec v'_i \\
		&= \vec L_c + \vec L' \\
\end{align*}
这样我们就把质点系的总动量分解成了质心的角动量与质点系相对角动量之和
\end{thm}
\begin{thm}[L \quad M\quad  J\quad  物理量关系]
\begin{align*}
		L = \int_{{}}^{{}} {M} \: d{t} {}\\
		L = J\omega \\
		M = J \alpha \\
		\vec r\times \vec F\\
\end{align*}

\end{thm}
%  ↑↑↑↑↑↑↑↑↑↑↑↑↑↑↑↑↑↑↑↑↑↑↑↑↑↑↑↑ 正文部分
\ifx\allfiles\undefined
\end{document}
\fi

