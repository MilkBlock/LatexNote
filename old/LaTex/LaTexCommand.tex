\documentclass[utf8]{ctexart}
\usepackage{geometry}
\usepackage{ulem}
\geometry{a4paper,scale=0.8}
\author{Mineral}
\title{LaTex常用命令}
\date {\today}
\begin{document}
		\maketitle
	    \section{文字}
	    \subsection{文字全局大小}
	    \par documentclass[12pt]{article}	
		\par 修改开头
		\subsection{文字局部大小}
		\begin{enumerate}
				\item tiny
				\item scriptsize
				\item footnotesize
				\item small
				\item normalsize
				\item large
				\item Large
				\item LARGE
				\item huge
				\item Huge									
		\end{enumerate}
		\par 这个大写字母区分层级的操作下饭!
		\par 注意局部模式是相比于全局字体的基础上变大变小。
	    \par \LARGE DEMO
	    \par \Large On SALE	
	    \par \huge Hello LaTex
		\normalsize
		\section{特殊符号}
		\subsection{删除线等}
		\par 首先加入 $\backslash$ usepackage{ulem}
		\begin{enumerate}
				\item $\backslash$ sout{文字} 删除线
				\item $\backslash$ uwave{文字} 波浪线	
				\item $\backslash$ xout{文字} 斜删除线
				\item $\backslash$ uuline{文字} 双下划线		
		\end{enumerate}
		\subsection{代码字体}
		\noindent	 
		使用verbatim 环境 

		\begin{verbatim}
		System.exit(0);
				
		\end{verbatim}
		\section{Math Expression}
		\begin{itemize}
				\item $\times $ xx
				\item $\ldots$ \ldots
				\item $\cdot $ **
				\item $\|A\| $ norm
				\item	$\delta \Delta \phi \varphi$
				\item $\to $

		\end{itemize}
		
\end{document}
