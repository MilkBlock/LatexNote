\ifx\allfiles\undefined
\documentclass[12pt, a4paper,oneside, UTF8]{ctexbook}
\usepackage[dvipsnames]{xcolor}
\usepackage{amsmath}   % 数学公式
\usepackage{amsthm}    % 定理环境
\usepackage{amssymb}   % 更多公式符号
\usepackage{graphicx}  % 插图
\usepackage{mathrsfs}  % 数学字体
\usepackage{enumitem}  % 列表
\usepackage{geometry}  % 页面调整
\usepackage{unicode-math}
\usepackage{ulem}
\usepackage{listings}
\usepackage{xcolor}
\lstset { %
		language=C++,
		backgroundcolor=\color{black!5}, % set backgroundcolor
		basicstyle=\footnotesize,% basic font setting
}
\usepackage[colorlinks,linkcolor=black]{hyperref}

\graphicspath{ {flg/},{../flg/}, {config/}, {../config/} }  % 配置图形文件检索目录
\linespread{1.5} % 行高

% 页码设置
\geometry{top=25.4mm,bottom=25.4mm,left=20mm,right=20mm,headheight=2.17cm,headsep=4mm,footskip=12mm}

% 设置列表环境的上下间距
\setenumerate[1]{itemsep=5pt,partopsep=0pt,parsep=\parskip,topsep=5pt}
\setitemize[1]{itemsep=5pt,partopsep=0pt,parsep=\parskip,topsep=5pt}
\setdescription{itemsep=5pt,partopsep=0pt,parsep=\parskip,topsep=5pt}

% 定理环境
% ########## 定理环境 start ####################################
\theoremstyle{definition}
\newtheorem{defn}{\indent 定义}[section]

\newtheorem{lemma}{\indent 引理}[section]    % 引理 定理 推论 准则 共用一个编号计数
\newtheorem{thm}[lemma]{\indent 定理}
\newtheorem{corollary}[lemma]{\indent 推论}
\newtheorem{criterion}[lemma]{\indent 准则}

\newtheorem{proposition}{\indent 命题}[section]
\newtheorem{example}{\indent \color{SeaGreen}{例}}[section] % 绿色文字的 例 ,不需要就去除\color{SeaGreen}{}
\newtheorem*{rmk}{\indent 注}

% 两种方式定义中文的 证明 和 解 的环境:
% 缺点:\qedhere 命令将会失效【技术有限,暂时无法解决】
\renewenvironment{proof}{\par\textbf{证明.}\;}{\qed\par}
\newenvironment{solution}{\par{\textbf{解.}}\;}{\qed\par}

% 缺点:\bf 是过时命令,可以用 textb f等替代,但编译会有关于字体的警告,不过不影响使用【技术有限,暂时无法解决】
%\renewcommand{\proofname}{\indent\bf 证明}
%\newenvironment{solution}{\begin{proof}[\indent\bf 解]}{\end{proof}}
% ######### 定理环境 end  #####################################

% ↓↓↓↓↓↓↓↓↓↓↓↓↓↓↓↓↓ 以下是自定义的命令  ↓↓↓↓↓↓↓↓↓↓↓↓↓↓↓↓

% 用于调整表格的高度  使用 \hline\xrowht{25pt}
\newcommand{\xrowht}[2][0]{\addstackgap[.5\dimexpr#2\relax]{\vphantom{#1}}}

% 表格环境内长内容换行
\newcommand{\tabincell}[2]{\begin{tabular}{@{}#1@{}}#2\end{tabular}}

% 使用\linespread{1.5} 之后 cases 环境的行高也会改变,重新定义一个 ca 环境可以自动控制 cases 环境行高
\newenvironment{ca}[1][1]{\linespread{#1} \selectfont \begin{cases}}{\end{cases}}
% 和上面一样
\newenvironment{vx}[1][1]{\linespread{#1} \selectfont \begin{vmatrix}}{\end{vmatrix}}

\def\d{\textup{d}} % 直立体 d 用于微分符号 dx
\def\R{\mathbb{R}} % 实数域
\newcommand{\bs}[1]{\boldsymbol{#1}}    % 加粗,常用于向量
\newcommand{\ora}[1]{\overrightarrow{#1}} % 向量

% 数学 平行 符号
\newcommand{\pll}{\kern 0.56em/\kern -0.8em /\kern 0.56em}

% 用于空行\myspace{1} 表示空一行 填 2 表示空两行  
\newcommand{\myspace}[1]{\par\vspace{#1\baselineskip}}

\begin{document}
% \title{{\Huge{\textbf{标题:一个 \LaTeX 的数学笔记模板}}} \\ ———副标题:格物致知,慎思明辨}
\author{作者:到底}
\date{\today}
\maketitle                   % 在单独的标题页上生成一个标题

\thispagestyle{empty}        % 前言页面不使用页码
\begin{center}
\Huge\textbf{前言}
\end{center}

if people do not believe that mathematics is simple, 
it is only because they do not realize how complicated life is. ——John von Neumann

\begin{flushright}
\begin{tabular}{c}
\today \\ 格物致知,慎思明辨
\end{tabular}
\end{flushright}

\newpage                      % 新的一页
\pagestyle{plain}             % 设置页眉和页脚的排版方式(plain:页眉是空的,页脚只包含一个居中的页码)
\setcounter{page}{1}          % 重新定义页码从第一页开始
\pagenumbering{Roman}         % 使用大写的罗马数字作为页码
\tableofcontents              % 生成目录

\newpage                      % 以下是正文
\pagestyle{plain}
\setcounter{page}{1}          % 使用阿拉伯数字作为页码
\pagenumbering{arabic}
% \setcounter{chapter}{-1}    % 设置 -1 可作为第零章绪论从第零章开始

\else
\fi

\chapter{演示}
\section{导数的概念}

\begin{defn}
设函数$y=f(x)$在点$x_0$的某个邻域内有定义,当自变量$x$在$x_0$处取得增量$\Delta{x}$(点$x_0+\Delta{x}$在该邻域内),因变量取得增量
\[ \Delta{y}=f(x_0+\Delta{x})-f(x_0) \]
如果$\Delta{y}$与$\Delta{x}$之比当$\Delta{x}\to 0$时的极限存在,那么称函数$y=f(x)$在点$x_0$处可导,并称这个极限为函数$y=f(x)$在$x_0$处的导数,记为$f'(x)$,即
\[
f'(x) = \lim_{\Delta{x}\to0}\frac{\Delta{y}}{\Delta{x}} = \lim_{\Delta{x}\to 0}\frac{f(x_0+\Delta{x})-f(x_0)}{\Delta{x}}
\]
或记为$y'|_{x=x_0},\;\dfrac{\d y}{\d x}\Big|_{x = x_0},\;\dfrac{\d f(x)}{\d x}\Big|_{x=x_0},\;f'(x) = \lim\limits_{x\to x_0}\dfrac{f(x)-f(x_0)}{x-x_0}$.
\end{defn}

\begin{lemma}[瞎编的引理]
好好学习 => 天天向上 
\[
		\left(\frac{\partial}{\partial \mathbf{{r}_{C}}} \left(\left(x\right)\mathbf{\hat{i}_{C}} + \left(y\right)\mathbf{\hat{j}_{C}} + \left(z\right)\mathbf{\hat{k}_{C}}\right)\right)\mathbf{\hat{i}_{C}} + \left(\frac{\frac{\partial}{\partial \mathbf{{\\\theta}_{C}}} \left(\left(x\right)\mathbf{\hat{i}_{C}} + \left(y\right)\mathbf{\hat{j}_{C}} + \left(z\right)\mathbf{\hat{k}_{C}}\right)}{\mathbf{{r}_{C}}}\right)\mathbf{\hat{j}_{C}} + \left(\frac{\partial}{\partial \mathbf{{z}_{C}}} \left(\left(x\right)\mathbf{\hat{i}_{C}} + \left(y\right)\mathbf{\hat{j}_{C}} + \left(z\right)\mathbf{\hat{k}_{C}}\right)\right)\mathbf{\hat{k}_{C}}
.\]  
\end{lemma}


\begin{thm}[离散函数的梯度的散度]
梯度的散度,在数学中的表达是Laplace 算子,对于离散函数$ f $,我们有
\begin{align*}
		\nabla^2 f &= \frac{f(x+\mathrm{d}x,y) + f(x-\mathrm{d}x,y) + f(x,y+\mathrm{d}y)+ f(x,y-\mathrm{d}y)  - 4f(x,y) }{\mathrm{d}x^2 }  \\
.\end{align*}
		因为根据泰勒展开我们有
	\begin{align*}
			f(x+\mathrm{d}x ,y) &=  f(x,y) + f'(x,y)\mathrm{d}x + f''(x,y) \mathrm{d}x^2 +\ldots\\
			f(x-\mathrm{d}x ,y) &=  f(x,y) + f'(x,y)\mathrm{d}x + f''(x,y) \mathrm{d}x^2 +\ldots\\
	.\end{align*}	
	两者相加即可得到
\end{thm}

\begin{corollary}[瞎编的推论]
你小子没点赞.
\end{corollary}

\begin{criterion}[夹逼准则]
$a(x) < b(x) < c(x)$,且$\lim{a} = \lim{c} = A$,那么$\lim{b} = A$
\end{criterion}

\begin{proposition}
差若毫厘,谬以千里.
\end{proposition}

\begin{example}
求 $ 1+2 = ?$
\end{example}

\begin{solution}
$\displaystyle 1 + 2 = (1+2)\int_0^1 x^2 \d x + (1+2)\int_0^1 x^2 \d x + (1+2)\int_0^1 x^2 \d x$
\end{solution}

\begin{rmk}
我不会
\end{rmk}

\begin{proof}
$1 + 2 = 2 + 1$.
\end{proof}

\section{偏导数}
\subsection{偏导数的定义及其计算法}
\begin{defn}
设函数$z=f(x,\,y)$在点$(x_0,\,y_0)$的某一邻域内有定义,当$y$固定在$y_0$而$x$在$x_0$处有增量$\Delta x$时,相应的函数有增量$f(x_0 + \Delta{x},\,y_0) - f(x_0,\,y_0)$,如果
\[
\lim_{\Delta{x} \to 0}\frac{f(x_0 + \Delta{x},\,y_0)-f(x_0,\,y_0)}{\Delta{x}}
\]
存在,那么称此极限为函数$z=f(x,\,y)$在点$(x_0,\,y_0)$处对$x$的偏导数(一点处的偏导),记作:
\[
\frac{\partial z}{\partial x}\Big| _{x=x_0,\, y=y_0} ,\enspace \frac{\partial f}{\partial x}\Big| _{x=x_0,\, y=y_0} ,\enspace f_x(x_0,y_0),\enskip {f_x}'(x_0,y_0)
\]
类似地,函数$z=f(x,\,y)$在点$(x_0,\,y_0)$处对$y$的偏导数定义为
\[
\lim_{\Delta{y} \to 0}\frac{f(x_0,\,y_0+\Delta{y})-f(x_0,\,y_0)}{\Delta{y}}
\]
记作同上,定义可推广到$n$元函数.
\end{defn}

\begin{example}
求$z = x^2\sin{2y}$的偏导数.
\end{example}
\begin{solution}
易得$\dfrac{\partial{z}}{\partial{x}} = 2x\sin{2y}+x^2(\sin{2y})' = 2x\sin{2y}$,$\dfrac{\partial{z}}{\partial{y}} = 2x^2\cos{2y}$.
\end{solution}

\myspace{1}

\begin{center}
{\LARGE 你小子记得点赞!}
\end{center}

\ifx\allfiles\undefined
\end{document}
\fi
